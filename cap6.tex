High-performance computing is the last order of the day. The trend in today's world is to keep improving and making the most of the potential of technology, in order to make science advance.
\vspace{2 mm}

Two current trends confirm this: the development of quantum computers and the installation of more supercomputers around the world and, perhaps the most unnoticed, the developments of coronavirus vaccines, discoveries and continuous studies in astronomy and time modeling for its prediction.
\vspace{2mm}

Implementing a high-performance computing system is relatively straightforward. Although there are various business solutions, it is possible to create your own system combining different free and open source tools, in exchange for the time it takes to implement and debug it, and the price to pay for the machines and their monthly energy maintenance.
\vspace{2mm}

Since maintaining multiple systems can be a tedious task, a number of good practices should be followed. Essentially, it can be summed up as system-related tasks must be done only once, regardless of how big the system is. For this reason, it is convenient to have all the resources centralized and shared with the different nodes, as well as to have in each node only what is necessary to exercise its function within the cluster.