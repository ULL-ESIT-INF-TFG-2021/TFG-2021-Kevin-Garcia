En este capítulo se realiza un presupuesto del trabajo realizado. Todas las herramientas utilizadas son gratuitas, por lo que el presupuesto necesario para este proyecto está definido por el material necesario y el sueldo del administrador de sistemas durante 4 meses (tiempo de duración de la asignatura).
\vspace{2mm}

Para el presupuesto de los servidores, se tomó de referencia una factura de la compra realizada por el departamento en 2014. Cabe a destacar que el precio del mismo modelo de servidor en el año de la realización de este trabajo (2021) es considerablemente más barato.

\section{Presupuesto}

\begin{center}
\begin{tabu} to 0.8\textwidth { | X[l] | X[l] | X[l] | }
 \hline
 \multicolumn{1}{|c|}{\bf Material} & \multicolumn{1}{|c|}{\bf Cantidad} & \multicolumn{1}{|c|}{\bf Presupuesto} \\
 \hline
 Servidor Dell PowerEdge R815 & \centering 3 & \centering 7.021€\\
  \hline
 Portatil de trabajo  & \centering 1 & \centering 800€\\
 \hline
   Total: &  & \centering 21.863€\\
 \hline
\end{tabu}
\end{center}

\begin{table}[htb]
   \centering
   \caption{Presupuesto del material}
   \label{chapter:presupuesto}
\end{table}

\begin{center}
\begin{tabu} to 0.8\textwidth { | X[l] | X[l] | X[l] | }
 \hline
 \multicolumn{1}{|c|}{\bf Personal} & \multicolumn{1}{|c|}{\bf Meses} & \multicolumn{1}{|c|}{\bf Presupuesto} \\
 \hline
 Ing. Informático & \centering 4 & \centering 1.200€ \\
  \hline
   Total: &  & \centering 4.800€\\
 \hline
\end{tabu}
\end{center}

\begin{table}[htb]
   \centering
   \caption{Presupuesto del personal}
   \label{chapter:presupuesto}
\end{table}

Para realizar este proyecto hicieron falta 4 meses y un precio final de 26.663€.